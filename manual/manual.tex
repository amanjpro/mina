\documentclass[10pt, a4paper]{article}

\usepackage[utf8]{inputenc}
\usepackage[english]{babel}
\usepackage[english]{isodate}
\usepackage[parfill]{parskip}
\usepackage[colorlinks=true]{hyperref}
\usepackage{xspace}
\usepackage{listings}


\lstdefinelanguage{scala}{
  morekeywords={abstract,case,catch,class,def,%
    do,else,extends,false,final,finally,%
    for,if,implicit,import,match,mixin,%
    new,null,object,override,package,%
    private,protected,requires,return,sealed,%
    super,this,throw,trait,true,try,%
    type,val,var,while,with,yield},
  otherkeywords={=>,<-,<\%,<:,>:,\#,@},
  sensitive=true,
  morecomment=[l]{//},
  morecomment=[n]{/*}{*/},
  morestring=[b]",
  morestring=[b]',
  morestring=[b]"""
}

\newcommand{\CT}{\texttt{CT}\xspace}
\newcommand{\RT}{\texttt{RT}\xspace}

\title{\textbf{\huge{Mina 1.0}}\\ \large{A Hybrid-Partial Evaluator for Scala}\\
  \small{\url{http://www.amanj.me/projects/mina}}}
\author{Amanj Sherwany \\ \small{\url{http://www.amanj.me}}}
\date{Draft of \today}

\begin{document}
\maketitle
\pagenumbering{arabic}

\tableofcontents
\newpage
\lstset{language=scala}
\section{Introduction}
\subsection{What is Mina?}
Mina is a Partial Evaluator for Scala. It is a port of Civet for Java, which is
an implementation of Hybrid-Partial Evaluator paper
(\url{http://www.cs.utexas.edu/~wcook/Civet/}). ``Hybrid partial evaluation
(HPE) is a pragmatic approach to partial evaluation that borrows ideas from
both online and offline partial evaluation. HPE performs offline-style
specialization using an online approach without static binding time analysis.
The goal of HPE is to provide a practical and predictable level of optimization
for programmers, with an implementation strategy that fits well within existing
compilers or interpreters. HPE requires the programmer to specify where partial
evaluation should be applied. It provides no termination guarantee and reports
errors in situations that violate simple binding time rules, or have incorrect
use of side effects in compile-time code.''\footnote{from Civet
\url{http://www.cs.utexas.edu/~wcook/Civet/}}

Mina is a Scala compiler plugin, runs after the \textbf{patmat} phase, and
partially evaluates Scala source code to a more efficient code, which has
exactly the same meaning as the original one.

The plugin name is mina and it is read like \emph{/mi:na:/}, which is a Kurdish
name of Forget-me-not flower \url{http://en.wikipedia.org/wiki/Forget-me-not}.


\subsection{Partial Evaluation}
``In computing, partial evaluation is a technique for several different types
of program optimization by specialization. The most straightforward application
is to produce new programs which run faster than the originals while being
guaranteed to behave in the same way.

A computer program, \emph{prog}, is seen as a mapping of input data into output
data:
\centerline{$prog: I_{static} \times I_{dynamic} \to O$}
$I_{static}$ the static data, is the part of the input data known at compile
time. The partial evaluator transforms $<prog, I_{static}>$ into $prog^{*}:
I_{dynamic} \to O$ by precompiling all static input at compile time. $prog^{*}$
is called the ``residual program'' and should run more efficiently than the
original program. The act of partial evaluation is said to ``residualize''
$prog$ to $prog^*$.''\footnote{from Wikipedia
\url{http://en.wikipedia.org/wiki/Partial_evaluation}}

\section{How to use Mina}

\subsection{Compile-time(\CT) vs. Runtmie(\RT)}
With Mina you can mark a block of code to be fully evaluated at compile time.
If a variable contains a compile time (\CT) code, then the variable itself is a
\CT variable and will not be present at runtime. Here is a simple of using \CT:

\begin{lstlisting}
  /**
   * 2 is evaluated at compile time,
   * therefore x is a compile-time variable,
   * i.e. it won't be present at runtime
   */
  val x = CT(2)    
  
  /**
   * y is a result of adding 5 to x,
   * but x is already a compile-time
   * variable, which means the result
   * of x + 5 will be a compile-time
   * value, which means y is also a
   * compile-time variable
   */              
  val y = x + 5    
  
  /**
   * since y is a compile-time variable,
   * this statement will be evaluated to 
   * println(7)
   */  
  println(y)
  
  /**
   * The above three statements will be 
   * evaluated to one statement only:
   * 
   * println(7)
   */
  
\end{lstlisting}

One can also mark a block of code to be not evaluated at all, simply by surrounding the block with \RT(...), for example:

\begin{lstlisting}
  /**
   * x won't be evaluated, since the 
   * right-hand side expression
   * is marked with RT
   */
  val x = RT(2)
  
  /**
   * y won't be evaluated too, 
   * since x is RT
   */
  val y = x + 5       
  
  /**
   * The following statement won't be 
   * evaluated too, since y is RT
   */
  println(y)
\end{lstlisting}

Now you should know that you can either tag an (expression/statement/block of
statements) is \CT or \RT. \CT makes sure that everything will be fully
evaluated at compile time, while \RT makes sure that everything will be
evaluated at runtime. In other words, no \CT inside \RT is allowed, and no \RT
inside \CT is allowed.



But if you do not tag an (expression/statement/block of statements) with
neither \CT nor \RT, then it will be partially evaluated at compilation-time
helping the program to speed-up. In the next sub-sections we will talk about
this in detail.


\section{Credits}

\end{document}