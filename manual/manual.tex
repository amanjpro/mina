\documentclass[10pt, a4paper]{article}

\usepackage[utf8]{inputenc}
\usepackage[english]{babel}
\usepackage[english]{isodate}
\usepackage[parfill]{parskip}
\usepackage[colorlinks=true]{hyperref}

\title{\textbf{\huge{Mina 1.0}}\\ \large{A Hybrid-Partial Evaluator for Scala}\\
  \small{\url{http://www.amanj.me/projects/mina}}}
\author{Amanj Sherwany \\ \small{\url{http://www.amanj.me}}}
\date{Draft of \today}

\begin{document}
\maketitle
\pagenumbering{arabic}

\tableofcontents
\newpage

\section{Introduction}
\subsection{What is Mina?}
Mina is a Partial Evaluator for Scala. It is a port of Civet for Java, which is
an implementation of Hybrid-Partial Evaluator paper
(\url{http://www.cs.utexas.edu/~wcook/Civet/}). ``Hybrid partial evaluation
(HPE) is a pragmatic approach to partial evaluation that borrows ideas from
both online and offline partial evaluation. HPE performs offline-style
specialization using an online approach without static binding time analysis.
The goal of HPE is to provide a practical and predictable level of optimization
for programmers, with an implementation strategy that fits well within existing
compilers or interpreters. HPE requires the programmer to specify where partial
evaluation should be applied. It provides no termination guarantee and reports
errors in situations that violate simple binding time rules, or have incorrect
use of side effects in compile-time code.''\footnote{from Civet
\url{http://www.cs.utexas.edu/~wcook/Civet/}}

Mina is a Scala compiler plugin, runs after the \textbf{patmat} phase, and
partially evaluates Scala source code to a more efficient code, which has
exactly the same meaning as the original one.

The plugin name is mina and it is read like \emph{/mi:na:/}, which is a Kurdish
name of Forget-me-not flower \url{http://en.wikipedia.org/wiki/Forget-me-not}.


\subsection{Partial Evaluation}
``In computing, partial evaluation is a technique for several different types
of program optimization by specialization. The most straightforward application
is to produce new programs which run faster than the originals while being
guaranteed to behave in the same way.

A computer program, \emph{prog}, is seen as a mapping of input data into output
data:
\centerline{$prog: I_{static} \times I_{dynamic} \to O$}
$I_{static}$ the static data, is the part of the input data known at compile
time. The partial evaluator transforms $<prog, I_{static}>$ into $prog^{*}:
I_{dynamic} \to O$ by precompiling all static input at compile time. $prog^{*}$
is called the ``residual program'' and should run more efficiently than the
original program. The act of partial evaluation is said to ``residualize''
$prog$ to $prog^*$.''\footnote{from Wikipedia
\url{http://en.wikipedia.org/wiki/Partial_evaluation}}

\section{How to use Mina}
\section{Use cases}
\section{Credits}

\end{document}